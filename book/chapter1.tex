

\chapter{Introduction}

This book is written for children who are interested in a focused education, focusing entirely on the math bits,
with exercises.  There will be strange and odd number games here and there, which are like puzzles, but with numbers.

The fundamental idea is that a student - a child - is able to understand plain mathematics, when taught clearly and
simply, without  attending to anything outside of the math itself. The author as a child thought his math books were
repetitive, slow, and boring, and he hopes to give a focused book for children who can do math, but don't want to be bored.

In particular, this book expects you to know how to count, how to read, and how to add. Not only that, this book will go FAST. If you're working through it and get stuck, that's fine. Go back and work on the parts that go into the hard part, and then try again, slower and more carefully.

A lot of the words, the ``terminology'', will be new and unusual.  Don't worry about that: they are just words, and words
can be learned.  We use the specific words here because they are the words that mathematicians use when they talk to
each other. There is a ``Glossary'' at the back which will define each word.

Happy Learning!

\section{For the Parents}

Hi!

This is a handbook I am working on for my child. I am not a curriculum expert, but a formerly homeschooled child with a Master's degree in Computer Science and a minor in bachelor's level mathematics. I am aiming to have this book be useful for 1:1 tutoring in a private after school or summer program with the general objective of completing the entire k-12 core subject matter as fast as the child can learn it. 

It attempts to eliminate all over trod review, extraneous pictures, and other "noise" from the learning process. It does have drills and other aspects of memorizion early on. Core knowledge is expected to be drilled into fluency. Inquiry elements will be included as well, to facilitate bridging the abstraction of numbers into the tangible daily life of children. 

Exercises are at the back of each major section. Generally I will aim to have 3x more exercises than I judge required for initial exposure, to facilitate extra review or practice. This is the conventional practice in collegiate texts, and I see no reason to abandon it. 

In addition, math is partially a practice based skill. Concepts and conceptual understanding are excellent, but being able to work the problem is the level required for mastery. 

If you are not familiar with the math being taught, do not fear: work through it on your own to a reasonable understanding level prior to sharing it with your child. 


If you have questions or want to ask the author something, you can email me at \textit{paul@nathan.house}.

Pleasant journeys to you!

\section{License, etc}

The source code for this book makes it freely available.  I would appreciate it if you did not print the book to sell. 

Please file issues you find at the following address: ...not set yet.

If you choose to contribute, your contributions are welcome, so long as they fit the philosophy outlined above. Please attempt to maintain a similar "voice".

\chapter{Arithmetic}

\section{Sets}
\subsection{Sets Themselves}
What is a Set?
A Set is a collection of objects.
The objects in a set are called elements or members of the set.
As an example, ``all apples on the counter'' is a set definition.
``All apples on the counter'' is the set, and each apple is an element of the set.

The set of all apples on the counter is denoted by $\{ \text{all apples on the counter} \}$.

As another example, the set of the age of first grades is denoted by $\{ 6, 7, 8 \}$.

Once you look, you can observe sets everywhere. The set of fruit in the house- the set of apples in the house and the set of strawberries. The set of food you won't eat - and the set of food you will. 

Sets are part of math, but they are kind of the ``frame'' around the early math picture. We'll return to them later in ``algebra'', so keep them in the pocket of your mind as we go along. 

Things you can do with multiple sets include:
- Intersection: The set of elements that are in both sets: the shared elements of both sets.
- Union: The set of elements that are in either set: adding everything in both sets to the same ``list''.
- Difference: The set of elements that are in one set but not the other.
- Complement: The set of elements that are not in the set.
- Count: how many elements are in the set? This has the special term "cardinality"- "what's the cardinality of a set" means "how many things are in it".

Examples: 

The set of your shirts intersects with the set of clothes you like to make the set of shirts you like. 

The set of food you like unions with the food you don't like to become the set of all food you tried. 

The set of shirts you have differences with the shirts you don't like... to result in the sets of short you do like. 

A complement is the "opposite", but requires yet another term, a ``superset''. The superset contains the set, and maybe more elements. 

So let's have an example: the Superset is all chess pieces in a chess game. One set is the White pieces. The complement of the set of White pieces is the set of Black pieces. 


If you notice some elements in a set, you can classify them as a ``subset'' of the first set. That's what we did in the chess example above. 

Question: if your Superset is all chess pieces, and if you have your Subset to be all chess pieces, what's the Complement? 

Answer: The Empty Set! It's the set with no elements. 

Notice that different numbers can be put into a set. 

Let's classify some numbers. This is, essentially, ``noticing patterns''; sometimes you are doing
something else, notice a behavior, and the reason for the behavior is an underlying pattern.

What groups of numbers can we classify numbers into?
The set of numbers that go 0, 2, 4, 6, 8, and so on are "even".
The set of numbers that go 1,3,5,7,9 and so on are called "odd". 
The set of numbers that start at 1 and go on forever are called natural numbers.
The set of numbers that starts at 0 and goes on forever are called whole numbers.
The set of numbers that includes negative numbers are called integers.

Wait, what's an \textit{integer}? An integer is a whole number that can be positive, negative, or zero.

Most of early arithmetic is done with natural numbers, whole numbers, and integers. (Arithmetic is what we're doing: the study of numbers and operations on numbers.)


What about numbers \textit{between} integers? These are called rational numbers: you can represent them as
a ``ratio'' between two numbers.
Don't worry about what a ratio means - we'll get to that later!

What about numbers that can't be represented as a ratio? These are called irrational numbers (There's a joke here, because rational also means thinking - are these the numbers that aren't thinking? Hmm. Math joke!).

And if you squish the ``irrational'' numbers together and the ``rational'' numbers together, you get the ``real'' numbers. (What about unreal numbers? We'll talk about THEM later).

These patterns are important because they help us understand the behavior of numbers. Each ``type'' of number has its own properties and behaviors.
Notice that, the natural numbers are a subset of the whole numbers, and the whole numbers are a subset of the integers.

If you have three cookies (an ODD number of cookies), and want to give an equal number of cookies to a friend, you can't, without breaking a cookie (very sad). Thats true of all odd numbers of cookies. But if the number of cookies you have is an EVEN number... you can share easily with one friend. 

Because sometimes math takes a lot of words to say things, and sometimes we just want to say it quickly, we use symbols to say what we mean. 

The set of all natural numbers is denoted by $\mathbb{N}$.

The set of all integers is denoted by
$\mathbb{Z}$.
The set of all rational numbers is denoted by $\mathbb{Q}$.
The set of all real numbers is denoted by $\mathbb{R}$.

There's not a special symbol for Odd and Even. Pretty...\textit{odd}, huh?

\subsection{Operations}

Operations means ''things we can do with stuff``.  You can TURN ON a car, RIDE a bike, COUNT numbers. 

The set operations we have are:

- intersection 
- union
- difference
- complement 
- cardinality 




\subsection{Exercises}

Using pencil and paper, write the answers:

The set of numbers from 1 to 10. 

Te set of numbers greater than 0 that you can add to get 10. 

The set of kinds of fruit in your house 

The set of fruit you like

The Intersection of fruit in your house and the fruit you like

The union of the set fruit you like and the set of fruit you don't like. 

The Cardinality of the set of Odd numbers less than 20

\subsection{Interesting Problems}

Can a set can have other sets as its elements? Why? Why not?

Let's suppose that a set can have other sets as its elements. Write down all the different sunsets of the set ${1}$, the. ${1,2}$, then ${1,2,3}$, then ${1,2,3,4}$. What do you notice? This "all possible sunsets of a set" operation is also known as the Power Set operation. 


\section{Ring}
In this section we will talk about the fundamental properties of typical numbers.
The term for this kind of set is a ``ring''.
\subsection{Addition}
\subsection{Multiplication}
\section{Inverses}
\section{Subtraction}
\section{Division}



\chapter{Glossary}
Arithmetic: The study of numbers and operations on numbers.
Complement: The set of elements that are not in the set.
Complex Numbers: Numbers that can be expressed in the form $a + bi$ where $a$ and $b$ are real numbers and $i$ is the imaginary unit.
Denoted: Represented by a symbol.
Difference: The set of elements that are in one set but not the other.
Element: An object in a set.
Integers: The set of all positive and negative integers.
Intersection: The set of elements that are in both sets.
Irrational Numbers: The set of all numbers that cannot be expressed as a fraction.
Natural Numbers: The set of all positive integers.
Rational Numbers: The set of all numbers that can be expressed as a fraction.
Real Numbers: All numbers that can be represented on a number line.
Ring: A set with two operations, addition and multiplication, that satisfy certain properties.
Set: A collection of objects.
Subset: A set that is entirely contained within another set.
Union: The set of elements that are in either set.
