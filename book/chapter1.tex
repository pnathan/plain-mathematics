

\chapter{Introduction}

This book is written for students who are interested a focused education, focusing entirely on the math bits,
with exercises.  There will be strange and odd number games here and there, which are like puzzles, but with numbers.

The fundamental idea is that a student - a child - is able to understand plain mathematics, when taught clearly and
simply, without  attending to anything outside of the math itself. The author as a child thought his math books were
repetitive, slow, and boring, and he hopes to give a focused book for children who can do math, but don't want to be bored.

A lot of the words, the ``terminology'', will be new and unusual.  Don't worry about that: they are just words, and words
can be learned.  We use the specific words here because they are the words that mathematicians use when they talk to
each other. There is a ``Glossary'' at the back which will define each word.

If you have questions or want to ask the author something, you can email him at \textit{paul@nathan.house}. He will be happy to help you.
If you're comfortable with the Internet, you can write him on the Github site where this book is written.

Happy learning!


\chapter{Arithmetic}

\section{Sets}
\subsection{Sets Themselves}
What is a Set?
A Set is a collection of objects.
The objects in a set are called elements or members of the set.
As an example, ``all apples on the counter'' is a set definition.
``All apples on the counter'' is the set, and each apple is an element of the set.

The set of all apples on the counter is denoted by $\{ \text{all apples on the counter} \}$.

As another example, the set of the age of first grades is denoted by $\{ 6, 7, 8 \}$.

Things you can do with multiple sets include:
- Intersection: The set of elements that are in both sets: the shared elements of both sets.
- Union: The set of elements that are in either set: adding everything in both sets to the same ``list''.
- Difference: The set of elements that are in one set but not the other.
- Complement: The set of elements that are not in the set.


As you see, different numbers can be put into a set. And, if you notice some elements in a set, you can classify them
as a ``subset'' of the first set.

Let's classify some numbers. This is, essentially, ``noticing patterns''; sometimes you are doing
something else, notice a behavior, and the reason for the behavior is an underlying pattern.

What groups of numbers can we classify numbers into?
The set of numbers that start at 1 and go on forever are called natural numbers.
The set of numbers that starts at 0 and goes on forever are called whole numbers.
The set of numbers that includes negative numbers are called integers.

Wait, what's an \textit{integer}? An integer is a whole number that can be positive, negative, or zero.

Most of early arithmetic is done with natural numbers, whole numbers, and integers. (Arithmetic is what we're doing: the study of numbers and operations on numbers.)


What about numbers \textit{between} integers? These are called rational numbers: you can represent them as
a ``ratio'' between two numbers.
Don't worry about what a ratio means - we'll get to that later!

What about numbers that can't be represented as a ratio? These are called irrational numbers (There's a joke here, because rational also means thinking - are these the numbers that aren't thinking? Hmm).

And if you squish the ``irrational'' numbers together and the ``rational'' numbers together, you get the ``real'' numbers.

These patterns are important because they help us understand the behavior of numbers. Each ``type'' of number has its own properties and behaviors.
Notice that, the natural numbers are a subset of the whole numbers, and the whole numbers are a subset of the integers.

The set of all natural numbers is denoted by $\mathbb{N}$, the set of all integers is denoted by
$\mathbb{Z}$, the set of all rational numbers is denoted by $\mathbb{Q}$,
the set of all real numbers is denoted by $\mathbb{R}$,
and the set of all complex numbers is denoted by $\mathbb{C}$.

\subsection{Operations}

\subsection{Exercises}

\section{Ring}
In this section we will talk about the fundamental properties of typical numbers.
The term for this kind of set is a ``ring''.
\subsection{Addition}
\subsection{Multiplication}
\section{Inverses}
\section{Subtraction}
\section{Division}



\chapter{Glossary}
Arithmetic: The study of numbers and operations on numbers.
Complement: The set of elements that are not in the set.
Complex Numbers: Numbers that can be expressed in the form $a + bi$ where $a$ and $b$ are real numbers and $i$ is the imaginary unit.
Denoted: Represented by a symbol.
Difference: The set of elements that are in one set but not the other.
Element: An object in a set.
Integers: The set of all positive and negative integers.
Intersection: The set of elements that are in both sets.
Irrational Numbers: The set of all numbers that cannot be expressed as a fraction.
Natural Numbers: The set of all positive integers.
Rational Numbers: The set of all numbers that can be expressed as a fraction.
Real Numbers: All numbers that can be represented on a number line.
Ring: A set with two operations, addition and multiplication, that satisfy certain properties.
Set: A collection of objects.
Subset: A set that is entirely contained within another set.
Union: The set of elements that are in either set.